\section{Politiche di sicurezza}
\subsection{Pratiche comuni}
Gli e-commerce, più di molti altri siti web, devono garantire un livello di sicurezza più che eccellente in quanto le conseguenze causate da eventuali bug o vulnerabilità sono potenzialmente catastrofiche per gli utenti che usufruiscono della piattaforma e gli amministratori che ne sono responsabili. Ho quindi adottato numerose pratiche comuni in ambito web per garantire un livello di sicurezza consono alla piattaforma, queste sono: 
\begin{itemize}
    \item \textbf{Rinominare i file caricati dagli utenti}, degli eventuali attaccanti possono caricare file con nomi particolari volti ad accedere a cartelle contenenti file sensibili che quindi risulterebbero compromessi; 
    \item \textbf{Controllare il contenuto dei file caricati dagli utenti}, gli unici file che gli utenti possono caricare sono immagini, viene quindi fatto un doppio controllo (lato client e lato server) per assicurarsi che i file caricati rispettino il corretto formato; 
    \item \textbf{Uso del pattern Post/Redirect/Get}, questo pattern particolare adottato in maniera estensiva nel web permette di ovviare al problema della ricarica o navigazione temporale delle pagine e richieste già inviate al server \cite{PRG}. Seguendo questo pattern tutte le richieste POST verranno prima elaborate e poi reindirizzate su la stessa o un'altra pagina che verrà ottenuta tramite richiesta GET, in questo modo l'utente nella sua navigazione non incontrerà mai messaggi relativi alla celebre "Conferma reinvio modulo"; 
    \item \textbf{HTTPS}, il sito usa il protocollo TLS/SSL insieme al protocollo HTTP per la navigazione sicura all'interno del sito, i dati trasmessi da e verso il sito sono quindi crittografati e non leggibili da possibili attaccanti; 
    \item \textbf{Autenticazione degli utenti}, gli utenti si devono autenticare per poter compiere azioni sul sito come inserire un prodotto; 
    \item \textbf{Password hashing}, le password usate dagli utenti non vengono salvate in chiaro dal database che invece le memorizza già crittografate, in questo modo in caso di data breach le password degli utenti anche se diffuse non sono leggibili.  
\end{itemize} 
\subsection{Autenticazione}
La piattaforma ha due modalità di accesso: il sito web e l'API. Entrambe le modalità devono effettuare l'autenticazione dell'utente che s'interfaccia con la piattaforma Rius.Co. Il sito web si occupa dell'autenticazione richiedendo agli utenti di impostare una password di lunghezza variabile tra 8 e 24 caratteri. I clienti possono anche richiedere la modifica della password che è comunque una pratica da fare periodicamente per evitare problemi relativi alla sicurezza del profilo. 
\medskip

L'API invece esegue l'autenticazione dell'utente in tutt'altro modo, gli utenti alla registrazione ricevono una chiave univoca privata (API key) di 32 byte che viene generata in modo casuale e non è modificabile. Questa chiave dovrà quindi essere inserita nel form di tutte le richieste fatte all'API per permettere l'identificazione dell'utente associato alla richiesta. Alcune richieste come ottenere un file in formato JSON contenente tutti i prodotti presenti sul Marketplace non necessitano dell'autenticazione e vengono eseguite anche senza l'API key in quanto sono informazioni pubbliche visualizzabili da chiunque. 
\bigskip

\textbf{Esempio di richiesta all'API tramite curl}
\begin{lstlisting}[style=dos]
$ curl -X GET --insecure "https://0.0.0.0:6066/Transactions/GetTransactionsByUserID/9" -F  "apiKey=aBQKfxy175kL1v5EdOwCkHWtgwAj9Kbzu339OkDIgxA="
\end{lstlisting}
\bigskip

\textbf{Risposta del server contenente tutte le transazioni in cui ha partecipato l'utente con l'API key indicata} 
\lstinputlisting[style=json]{content/code/response.json}
\subsection{Password hashing} 
